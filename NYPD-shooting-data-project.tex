% Options for packages loaded elsewhere
\PassOptionsToPackage{unicode}{hyperref}
\PassOptionsToPackage{hyphens}{url}
%
\documentclass[
]{article}
\usepackage{amsmath,amssymb}
\usepackage{lmodern}
\usepackage{iftex}
\ifPDFTeX
  \usepackage[T1]{fontenc}
  \usepackage[utf8]{inputenc}
  \usepackage{textcomp} % provide euro and other symbols
\else % if luatex or xetex
  \usepackage{unicode-math}
  \defaultfontfeatures{Scale=MatchLowercase}
  \defaultfontfeatures[\rmfamily]{Ligatures=TeX,Scale=1}
\fi
% Use upquote if available, for straight quotes in verbatim environments
\IfFileExists{upquote.sty}{\usepackage{upquote}}{}
\IfFileExists{microtype.sty}{% use microtype if available
  \usepackage[]{microtype}
  \UseMicrotypeSet[protrusion]{basicmath} % disable protrusion for tt fonts
}{}
\makeatletter
\@ifundefined{KOMAClassName}{% if non-KOMA class
  \IfFileExists{parskip.sty}{%
    \usepackage{parskip}
  }{% else
    \setlength{\parindent}{0pt}
    \setlength{\parskip}{6pt plus 2pt minus 1pt}}
}{% if KOMA class
  \KOMAoptions{parskip=half}}
\makeatother
\usepackage{xcolor}
\usepackage[margin=1in]{geometry}
\usepackage{color}
\usepackage{fancyvrb}
\newcommand{\VerbBar}{|}
\newcommand{\VERB}{\Verb[commandchars=\\\{\}]}
\DefineVerbatimEnvironment{Highlighting}{Verbatim}{commandchars=\\\{\}}
% Add ',fontsize=\small' for more characters per line
\usepackage{framed}
\definecolor{shadecolor}{RGB}{248,248,248}
\newenvironment{Shaded}{\begin{snugshade}}{\end{snugshade}}
\newcommand{\AlertTok}[1]{\textcolor[rgb]{0.94,0.16,0.16}{#1}}
\newcommand{\AnnotationTok}[1]{\textcolor[rgb]{0.56,0.35,0.01}{\textbf{\textit{#1}}}}
\newcommand{\AttributeTok}[1]{\textcolor[rgb]{0.77,0.63,0.00}{#1}}
\newcommand{\BaseNTok}[1]{\textcolor[rgb]{0.00,0.00,0.81}{#1}}
\newcommand{\BuiltInTok}[1]{#1}
\newcommand{\CharTok}[1]{\textcolor[rgb]{0.31,0.60,0.02}{#1}}
\newcommand{\CommentTok}[1]{\textcolor[rgb]{0.56,0.35,0.01}{\textit{#1}}}
\newcommand{\CommentVarTok}[1]{\textcolor[rgb]{0.56,0.35,0.01}{\textbf{\textit{#1}}}}
\newcommand{\ConstantTok}[1]{\textcolor[rgb]{0.00,0.00,0.00}{#1}}
\newcommand{\ControlFlowTok}[1]{\textcolor[rgb]{0.13,0.29,0.53}{\textbf{#1}}}
\newcommand{\DataTypeTok}[1]{\textcolor[rgb]{0.13,0.29,0.53}{#1}}
\newcommand{\DecValTok}[1]{\textcolor[rgb]{0.00,0.00,0.81}{#1}}
\newcommand{\DocumentationTok}[1]{\textcolor[rgb]{0.56,0.35,0.01}{\textbf{\textit{#1}}}}
\newcommand{\ErrorTok}[1]{\textcolor[rgb]{0.64,0.00,0.00}{\textbf{#1}}}
\newcommand{\ExtensionTok}[1]{#1}
\newcommand{\FloatTok}[1]{\textcolor[rgb]{0.00,0.00,0.81}{#1}}
\newcommand{\FunctionTok}[1]{\textcolor[rgb]{0.00,0.00,0.00}{#1}}
\newcommand{\ImportTok}[1]{#1}
\newcommand{\InformationTok}[1]{\textcolor[rgb]{0.56,0.35,0.01}{\textbf{\textit{#1}}}}
\newcommand{\KeywordTok}[1]{\textcolor[rgb]{0.13,0.29,0.53}{\textbf{#1}}}
\newcommand{\NormalTok}[1]{#1}
\newcommand{\OperatorTok}[1]{\textcolor[rgb]{0.81,0.36,0.00}{\textbf{#1}}}
\newcommand{\OtherTok}[1]{\textcolor[rgb]{0.56,0.35,0.01}{#1}}
\newcommand{\PreprocessorTok}[1]{\textcolor[rgb]{0.56,0.35,0.01}{\textit{#1}}}
\newcommand{\RegionMarkerTok}[1]{#1}
\newcommand{\SpecialCharTok}[1]{\textcolor[rgb]{0.00,0.00,0.00}{#1}}
\newcommand{\SpecialStringTok}[1]{\textcolor[rgb]{0.31,0.60,0.02}{#1}}
\newcommand{\StringTok}[1]{\textcolor[rgb]{0.31,0.60,0.02}{#1}}
\newcommand{\VariableTok}[1]{\textcolor[rgb]{0.00,0.00,0.00}{#1}}
\newcommand{\VerbatimStringTok}[1]{\textcolor[rgb]{0.31,0.60,0.02}{#1}}
\newcommand{\WarningTok}[1]{\textcolor[rgb]{0.56,0.35,0.01}{\textbf{\textit{#1}}}}
\usepackage{graphicx}
\makeatletter
\def\maxwidth{\ifdim\Gin@nat@width>\linewidth\linewidth\else\Gin@nat@width\fi}
\def\maxheight{\ifdim\Gin@nat@height>\textheight\textheight\else\Gin@nat@height\fi}
\makeatother
% Scale images if necessary, so that they will not overflow the page
% margins by default, and it is still possible to overwrite the defaults
% using explicit options in \includegraphics[width, height, ...]{}
\setkeys{Gin}{width=\maxwidth,height=\maxheight,keepaspectratio}
% Set default figure placement to htbp
\makeatletter
\def\fps@figure{htbp}
\makeatother
\setlength{\emergencystretch}{3em} % prevent overfull lines
\providecommand{\tightlist}{%
  \setlength{\itemsep}{0pt}\setlength{\parskip}{0pt}}
\setcounter{secnumdepth}{-\maxdimen} % remove section numbering
\ifLuaTeX
  \usepackage{selnolig}  % disable illegal ligatures
\fi
\IfFileExists{bookmark.sty}{\usepackage{bookmark}}{\usepackage{hyperref}}
\IfFileExists{xurl.sty}{\usepackage{xurl}}{} % add URL line breaks if available
\urlstyle{same} % disable monospaced font for URLs
\hypersetup{
  pdftitle={NYPD shooting data project},
  pdfauthor={Kyle Ritland},
  hidelinks,
  pdfcreator={LaTeX via pandoc}}

\title{NYPD shooting data project}
\author{Kyle Ritland}
\date{2023-02-21}

\begin{document}
\maketitle

\hypertarget{what-questions-do-you-hope-to-answer}{%
\subsection{What questions do you hope to
answer}\label{what-questions-do-you-hope-to-answer}}

In this paper I hope to answer three questions:

\begin{enumerate}
\def\labelenumi{\arabic{enumi}.}
\item
  How have the number of shooting incidents changed over time? Have they
  changed periodically, have they remained constant over time, have they
  been increasing or decreasing? Has the same been true for murders?
\item
  Are shooting rates a predictor of murder rates?
\item
  Have particular ethnic groups been the target of shooting incidents?
  Has one group been the victim more often than others, have different
  groups become the majority victimized group over time or has it been
  roughly consistently one group over time?
\end{enumerate}

\hypertarget{why-are-these-questions-important}{%
\subsection{Why are these questions
important?}\label{why-are-these-questions-important}}

In the United States, a multitude of different racial groups make up the
population, and it is important to investigate if or if not all racial
and ethnic groups are treated equally, exposed to the same environments
that every other racial group is, are at the same level of finanacial
security as other groups. For instance, from U.S. census data, we know
that the average net worth of a house with a black homeowener was around
\$10,000, while for a non-hispanic white homeowner, the average was over
\$170,000, a 1,700\% difference in net worth
(\url{https://www.census.gov/library/stories/2021/09/understanding-equity-through-census-bureau-data.html}).

With such stark differences between groups existing in the present day,
it is important for american citizens to be aware of such differences so
that they can affect change in their communities and their country. Many
communities across the U.S. are just as racially and ethnically diverse
as the country as a whole, sometimes more so. For citizens to better
their communities, they have to know what areas of their communities
need the most attention, and in many cases, those areas tend to focused
on groups of a particular racial identity.

\hypertarget{libraries-to-import}{%
\subsection{Libraries to import}\label{libraries-to-import}}

\begin{Shaded}
\begin{Highlighting}[]
\FunctionTok{library}\NormalTok{(tidyverse) }
\end{Highlighting}
\end{Shaded}

\begin{verbatim}
## -- Attaching packages --------------------------------------- tidyverse 1.3.2 --
## v ggplot2 3.4.0     v purrr   1.0.1
## v tibble  3.1.8     v dplyr   1.1.0
## v tidyr   1.3.0     v stringr 1.5.0
## v readr   2.1.3     v forcats 1.0.0
## -- Conflicts ------------------------------------------ tidyverse_conflicts() --
## x dplyr::filter() masks stats::filter()
## x dplyr::lag()    masks stats::lag()
\end{verbatim}

\begin{Shaded}
\begin{Highlighting}[]
\FunctionTok{library}\NormalTok{(ggplot2)}
\FunctionTok{library}\NormalTok{(lubridate)}
\end{Highlighting}
\end{Shaded}

\begin{verbatim}
## 
## Attaching package: 'lubridate'
## 
## The following objects are masked from 'package:base':
## 
##     date, intersect, setdiff, union
\end{verbatim}

\hypertarget{data-to-get}{%
\subsection{Data to get}\label{data-to-get}}

\begin{Shaded}
\begin{Highlighting}[]
\NormalTok{url\_in }\OtherTok{\textless{}{-}} \StringTok{"https://data.cityofnewyork.us/api/views/833y{-}fsy8/rows.csv?accessType=DOWNLOAD"}
\end{Highlighting}
\end{Shaded}

\begin{Shaded}
\begin{Highlighting}[]
\NormalTok{data }\OtherTok{\textless{}{-}} \FunctionTok{read\_csv}\NormalTok{(url\_in)}
\end{Highlighting}
\end{Shaded}

\begin{verbatim}
## Rows: 25596 Columns: 19
## -- Column specification --------------------------------------------------------
## Delimiter: ","
## chr  (10): OCCUR_DATE, BORO, LOCATION_DESC, PERP_AGE_GROUP, PERP_SEX, PERP_R...
## dbl   (7): INCIDENT_KEY, PRECINCT, JURISDICTION_CODE, X_COORD_CD, Y_COORD_CD...
## lgl   (1): STATISTICAL_MURDER_FLAG
## time  (1): OCCUR_TIME
## 
## i Use `spec()` to retrieve the full column specification for this data.
## i Specify the column types or set `show_col_types = FALSE` to quiet this message.
\end{verbatim}

\hypertarget{preparing-data-for-analysis}{%
\subsection{Preparing data for
analysis}\label{preparing-data-for-analysis}}

\hypertarget{clean-data-for-question-1-and-2}{%
\subsubsection{Clean data for question 1 and
2}\label{clean-data-for-question-1-and-2}}

\begin{enumerate}
\def\labelenumi{\arabic{enumi})}
\tightlist
\item
  Group data by date, convert column `STATISTICAL\_MURDER\_FLAG' into
  numerical format for later summation and create columns for year,
  month and date.
\end{enumerate}

\begin{Shaded}
\begin{Highlighting}[]
\NormalTok{data\_cln }\OtherTok{\textless{}{-}}\NormalTok{ data }\SpecialCharTok{\%\textgreater{}\%} \FunctionTok{group\_by}\NormalTok{(OCCUR\_DATE) }\SpecialCharTok{\%\textgreater{}\%} \FunctionTok{summarize}\NormalTok{(}\AttributeTok{Num\_shootings =} \FunctionTok{sum}\NormalTok{(STATISTICAL\_MURDER\_FLAG) }\SpecialCharTok{+} \FunctionTok{sum}\NormalTok{(}\SpecialCharTok{!}\NormalTok{STATISTICAL\_MURDER\_FLAG), }\AttributeTok{Num\_murders =} \FunctionTok{sum}\NormalTok{(STATISTICAL\_MURDER\_FLAG),) }\SpecialCharTok{\%\textgreater{}\%} \FunctionTok{separate}\NormalTok{(}\StringTok{"OCCUR\_DATE"}\NormalTok{, }\FunctionTok{c}\NormalTok{(}\StringTok{"Month"}\NormalTok{, }\StringTok{"Day"}\NormalTok{, }\StringTok{"Year"}\NormalTok{), }\AttributeTok{sep =} \StringTok{"/"}\NormalTok{,}\AttributeTok{remove =} \ConstantTok{FALSE}\NormalTok{) }\SpecialCharTok{\%\textgreater{}\%} \FunctionTok{ungroup}\NormalTok{()}
\end{Highlighting}
\end{Shaded}

\begin{enumerate}
\def\labelenumi{\arabic{enumi})}
\setcounter{enumi}{1}
\tightlist
\item
  Create a new column comprised of only year and month.
\end{enumerate}

\begin{Shaded}
\begin{Highlighting}[]
\NormalTok{data\_cln\_y\_m }\OtherTok{\textless{}{-}}\NormalTok{ data\_cln }\SpecialCharTok{\%\textgreater{}\%} \FunctionTok{mutate}\NormalTok{(}\AttributeTok{Year\_Month =} \FunctionTok{as.Date}\NormalTok{(}\FunctionTok{with}\NormalTok{(data\_cln,}\FunctionTok{paste}\NormalTok{(Year,Month,}\StringTok{"01"}\NormalTok{,}\AttributeTok{sep=}\StringTok{"{-}"}\NormalTok{)),}\AttributeTok{format=}\StringTok{"\%Y{-}\%m{-}\%d"}\NormalTok{), )}
\end{Highlighting}
\end{Shaded}

\begin{enumerate}
\def\labelenumi{\arabic{enumi})}
\setcounter{enumi}{2}
\tightlist
\item
  Group data by the new year and month column
\end{enumerate}

\begin{Shaded}
\begin{Highlighting}[]
\NormalTok{data\_cln\_y\_m }\OtherTok{\textless{}{-}}\NormalTok{ data\_cln\_y\_m }\SpecialCharTok{\%\textgreater{}\%} \FunctionTok{group\_by}\NormalTok{(Year\_Month) }\SpecialCharTok{\%\textgreater{}\%} \FunctionTok{summarize}\NormalTok{(}\AttributeTok{Num\_shootings =} \FunctionTok{sum}\NormalTok{(Num\_shootings), }\AttributeTok{Num\_murders =} \FunctionTok{sum}\NormalTok{(Num\_murders)) }\SpecialCharTok{\%\textgreater{}\%} \FunctionTok{ungroup}\NormalTok{()}
\end{Highlighting}
\end{Shaded}

\hypertarget{clean-data-for-question-3}{%
\subsubsection{Clean data for question
3}\label{clean-data-for-question-3}}

\begin{enumerate}
\def\labelenumi{\arabic{enumi})}
\tightlist
\item
  Group data by date and recorded victim race, convert column
  `STATISTICAL\_MURDER\_FLAG' into numerical format for later summation
  and create columns for year, month and date.
\end{enumerate}

\begin{Shaded}
\begin{Highlighting}[]
\NormalTok{ethnicities }\OtherTok{=} \FunctionTok{sort}\NormalTok{(}\FunctionTok{unique}\NormalTok{(data}\SpecialCharTok{$}\NormalTok{VIC\_RACE))}

\NormalTok{data\_cln\_eth }\OtherTok{\textless{}{-}}\NormalTok{ data }\SpecialCharTok{\%\textgreater{}\%} \FunctionTok{group\_by}\NormalTok{(OCCUR\_DATE, VIC\_RACE) }\SpecialCharTok{\%\textgreater{}\%} \FunctionTok{summarize}\NormalTok{(}\AttributeTok{Num\_shootings =} \FunctionTok{sum}\NormalTok{(STATISTICAL\_MURDER\_FLAG) }\SpecialCharTok{+} \FunctionTok{sum}\NormalTok{(}\SpecialCharTok{!}\NormalTok{STATISTICAL\_MURDER\_FLAG), }\AttributeTok{Num\_murders =} \FunctionTok{sum}\NormalTok{(STATISTICAL\_MURDER\_FLAG),) }\SpecialCharTok{\%\textgreater{}\%} \FunctionTok{separate}\NormalTok{(}\StringTok{"OCCUR\_DATE"}\NormalTok{, }\FunctionTok{c}\NormalTok{(}\StringTok{"Month"}\NormalTok{, }\StringTok{"Day"}\NormalTok{, }\StringTok{"Year"}\NormalTok{), }\AttributeTok{sep =} \StringTok{"/"}\NormalTok{,}\AttributeTok{remove =} \ConstantTok{FALSE}\NormalTok{) }\SpecialCharTok{\%\textgreater{}\%} \FunctionTok{ungroup}\NormalTok{()}
\end{Highlighting}
\end{Shaded}

\begin{verbatim}
## `summarise()` has grouped output by 'OCCUR_DATE'. You can override using the
## `.groups` argument.
\end{verbatim}

\begin{enumerate}
\def\labelenumi{\arabic{enumi})}
\setcounter{enumi}{1}
\tightlist
\item
  Create a new column comprised of only year and month.
\end{enumerate}

\begin{Shaded}
\begin{Highlighting}[]
\NormalTok{data\_cln\_eth }\OtherTok{\textless{}{-}}\NormalTok{ data\_cln\_eth }\SpecialCharTok{\%\textgreater{}\%} \FunctionTok{mutate}\NormalTok{(}\AttributeTok{Year\_Month =} \FunctionTok{as.Date}\NormalTok{(}\FunctionTok{with}\NormalTok{(data\_cln\_eth,}\FunctionTok{paste}\NormalTok{(Year,Month,}\StringTok{"01"}\NormalTok{,}\AttributeTok{sep=}\StringTok{"{-}"}\NormalTok{)),}\AttributeTok{format=}\StringTok{"\%Y{-}\%m{-}\%d"}\NormalTok{), )}
\end{Highlighting}
\end{Shaded}

\begin{enumerate}
\def\labelenumi{\arabic{enumi})}
\setcounter{enumi}{2}
\tightlist
\item
  Group data by the new year/month column and victim raze column
\end{enumerate}

\begin{Shaded}
\begin{Highlighting}[]
\NormalTok{data\_cln\_eth\_y\_m }\OtherTok{\textless{}{-}}\NormalTok{ data\_cln\_eth }\SpecialCharTok{\%\textgreater{}\%} \FunctionTok{group\_by}\NormalTok{(Year\_Month, VIC\_RACE) }\SpecialCharTok{\%\textgreater{}\%} \FunctionTok{summarize}\NormalTok{( }\AttributeTok{Num\_shootings =} \FunctionTok{sum}\NormalTok{(Num\_shootings), }\AttributeTok{Num\_murders =} \FunctionTok{sum}\NormalTok{(Num\_murders)) }\SpecialCharTok{\%\textgreater{}\%} \FunctionTok{ungroup}\NormalTok{()}
\end{Highlighting}
\end{Shaded}

\begin{verbatim}
## `summarise()` has grouped output by 'Year_Month'. You can override using the
## `.groups` argument.
\end{verbatim}

\begin{enumerate}
\def\labelenumi{\arabic{enumi}.}
\setcounter{enumi}{3}
\tightlist
\item
  Combine the data frame from question 1 with the new data frame, so as
  to pair the total number of shootings and murders with the number of
  shootings and murders for each victim race category.
\end{enumerate}

\begin{Shaded}
\begin{Highlighting}[]
\NormalTok{data\_cln\_eth\_y\_m }\OtherTok{\textless{}{-}} \FunctionTok{merge}\NormalTok{(data\_cln\_eth\_y\_m, data\_cln\_y\_m, }\AttributeTok{by=}\FunctionTok{c}\NormalTok{(}\StringTok{"Year\_Month"}\NormalTok{)) }\SpecialCharTok{\%\textgreater{}\%} \FunctionTok{rename}\NormalTok{(}\StringTok{"Num\_shootings\_eth"} \OtherTok{=} \StringTok{"Num\_shootings.x"}\NormalTok{, }\StringTok{"Num\_shootings\_tot"} \OtherTok{=} \StringTok{"Num\_shootings.y"}\NormalTok{,}\StringTok{"Num\_murders\_eth"} \OtherTok{=} \StringTok{"Num\_murders.x"}\NormalTok{, }\StringTok{"Num\_murders\_tot"} \OtherTok{=} \StringTok{"Num\_murders.y"}\NormalTok{)}
\end{Highlighting}
\end{Shaded}

\begin{enumerate}
\def\labelenumi{\arabic{enumi}.}
\setcounter{enumi}{4}
\tightlist
\item
  Use the columns of shootings and murders for each ethnicity category
  with the total number of shootings and murders to determine the
  percentage of the data each ethnicity makes up of the total data.
\end{enumerate}

\begin{Shaded}
\begin{Highlighting}[]
\NormalTok{data\_cln\_eth\_y\_m }\OtherTok{\textless{}{-}}\NormalTok{ data\_cln\_eth\_y\_m }\SpecialCharTok{\%\textgreater{}\%} \FunctionTok{mutate}\NormalTok{(}\AttributeTok{shootings\_perc =}\NormalTok{ Num\_shootings\_eth}\SpecialCharTok{/}\NormalTok{Num\_shootings\_tot, }\AttributeTok{murders\_perc =}\NormalTok{ Num\_murders\_eth}\SpecialCharTok{/}\NormalTok{Num\_murders\_tot)}
\end{Highlighting}
\end{Shaded}

\begin{enumerate}
\def\labelenumi{\arabic{enumi}.}
\setcounter{enumi}{5}
\tightlist
\item
  Create a new data frame to record the overall percentage that each
  ethnic group constitutes of the total data.
\end{enumerate}

\begin{Shaded}
\begin{Highlighting}[]
\NormalTok{data\_cln\_eth\_tots }\OtherTok{\textless{}{-}}\NormalTok{ data\_cln\_eth }\SpecialCharTok{\%\textgreater{}\%} \FunctionTok{group\_by}\NormalTok{(VIC\_RACE) }\SpecialCharTok{\%\textgreater{}\%} \FunctionTok{summarize}\NormalTok{( }\AttributeTok{Num\_shootings =} \FunctionTok{sum}\NormalTok{(Num\_shootings), }\AttributeTok{Num\_murders =} \FunctionTok{sum}\NormalTok{(Num\_murders)) }\SpecialCharTok{\%\textgreater{}\%} \FunctionTok{mutate}\NormalTok{(}\AttributeTok{Num\_shootings\_tot =} \FunctionTok{rep}\NormalTok{(}\FunctionTok{sum}\NormalTok{(data\_cln\_y\_m}\SpecialCharTok{$}\NormalTok{Num\_shootings), }\AttributeTok{length.out =} \FunctionTok{length}\NormalTok{(ethnicities)), }\AttributeTok{Num\_murders\_tot =} \FunctionTok{rep}\NormalTok{(}\FunctionTok{sum}\NormalTok{(data\_cln\_y\_m}\SpecialCharTok{$}\NormalTok{Num\_murders), }\AttributeTok{length.out =} \FunctionTok{length}\NormalTok{(ethnicities))) }\SpecialCharTok{\%\textgreater{}\%} \FunctionTok{ungroup}\NormalTok{()}
\end{Highlighting}
\end{Shaded}

\begin{Shaded}
\begin{Highlighting}[]
\NormalTok{data\_cln\_eth\_tots }\OtherTok{\textless{}{-}}\NormalTok{ data\_cln\_eth\_tots }\SpecialCharTok{\%\textgreater{}\%} \FunctionTok{mutate}\NormalTok{(}\AttributeTok{shootings\_perc =}\NormalTok{ Num\_shootings}\SpecialCharTok{/}\NormalTok{Num\_shootings\_tot, }\AttributeTok{murders\_perc =}\NormalTok{ Num\_murders}\SpecialCharTok{/}\NormalTok{Num\_murders\_tot)}
\end{Highlighting}
\end{Shaded}

\hypertarget{model-for-question-2}{%
\subsection{Model for question 2}\label{model-for-question-2}}

Next I will make a linear model of murders as a function of shooting
incidents, to see if there is a statistically significant correlation
between shooting incident rates and murder rates. Then, I will add that
data to the original data frame for use in graphing.

\begin{Shaded}
\begin{Highlighting}[]
\NormalTok{model }\OtherTok{\textless{}{-}} \FunctionTok{lm}\NormalTok{(Num\_murders }\SpecialCharTok{\textasciitilde{}}\NormalTok{ Num\_shootings, data\_cln\_y\_m)}

\NormalTok{nyc\_final\_data }\OtherTok{\textless{}{-}}\NormalTok{ data\_cln\_y\_m }\SpecialCharTok{\%\textgreater{}\%} \FunctionTok{ungroup}\NormalTok{() }\SpecialCharTok{\%\textgreater{}\%} \FunctionTok{mutate}\NormalTok{( }\AttributeTok{murders\_pred=} \FunctionTok{predict}\NormalTok{(model))}
\end{Highlighting}
\end{Shaded}

\hypertarget{analysis}{%
\subsection{Analysis}\label{analysis}}

\hypertarget{analysis-of-question-1}{%
\subsubsection{Analysis of Question 1}\label{analysis-of-question-1}}

Here is a graph of the total shooting incidents over the span of data
collection.

\begin{Shaded}
\begin{Highlighting}[]
\NormalTok{shootings\_plot }\OtherTok{\textless{}{-}}\NormalTok{ data\_cln\_y\_m }\SpecialCharTok{\%\textgreater{}\%} \FunctionTok{ggplot}\NormalTok{(}\FunctionTok{aes}\NormalTok{(}\AttributeTok{x =}\NormalTok{ Year\_Month)) }\SpecialCharTok{+} \FunctionTok{geom\_area}\NormalTok{( }\FunctionTok{aes}\NormalTok{(}\AttributeTok{y =}\NormalTok{ Num\_shootings), }\AttributeTok{fill=}\StringTok{"grey50"}\NormalTok{, }\AttributeTok{color =} \DecValTok{1}\NormalTok{, }\AttributeTok{lwd =} \FloatTok{0.5}\NormalTok{, }\AttributeTok{linetype =} \DecValTok{1}\NormalTok{) }\SpecialCharTok{+} \FunctionTok{scale\_x\_date}\NormalTok{(}\AttributeTok{date\_labels =} \StringTok{"\%Y{-}\%b"}\NormalTok{, }\AttributeTok{date\_breaks =} \StringTok{"2 year"}\NormalTok{) }\SpecialCharTok{+} \FunctionTok{scale\_y\_continuous}\NormalTok{(}\AttributeTok{limits =} \FunctionTok{c}\NormalTok{(}\DecValTok{0}\NormalTok{,}\DecValTok{350}\NormalTok{)) }\SpecialCharTok{+} \FunctionTok{theme}\NormalTok{(}\AttributeTok{axis.text.x =} \FunctionTok{element\_text}\NormalTok{(}\AttributeTok{angle =} \DecValTok{45}\NormalTok{, }\AttributeTok{vjust =} \DecValTok{1}\NormalTok{, }\AttributeTok{hjust=}\DecValTok{1}\NormalTok{)) }\SpecialCharTok{+} \FunctionTok{labs}\NormalTok{(}\AttributeTok{title=}\StringTok{"Number of shooting incidents per month,}\SpecialCharTok{\textbackslash{}n}\StringTok{Jan{-}2006 to Dec{-}2022"}\NormalTok{, }\AttributeTok{x=} \StringTok{"Year{-}Month"}\NormalTok{, }\AttributeTok{y =} \StringTok{"Number of shooting incidents,}\SpecialCharTok{\textbackslash{}n}\StringTok{per month"}\NormalTok{)}
\end{Highlighting}
\end{Shaded}

\begin{verbatim}
## Warning: Using `size` aesthetic for lines was deprecated in ggplot2 3.4.0.
## i Please use `linewidth` instead.
\end{verbatim}

\begin{Shaded}
\begin{Highlighting}[]
\FunctionTok{plot}\NormalTok{(shootings\_plot)}
\end{Highlighting}
\end{Shaded}

\includegraphics{NYPD-shooting-data-project_files/figure-latex/plot_1-1.pdf}

Interestingly, there does appear to be a cyclical nature to shooting
incident rates in NYC over the data collection period. If you look at
the peaks and troughs, every peak resides in summer months, while
troughs reside in winter months, and shooting incident numbers
increasing into the summer peak and decreasing into the winter trough.
Every low point of a trough lies on the month of January or February. In
addition, shooting rates have been decreasing over time, and decreasing
in discrete steps, up until 2020, when the COVID-19 pandemic started.

Now lets look and see if the murder rates in NYC also showed the same
seasonal pattern as overall shooting incidents. Below is a graph showing
the number of shooting incidents that resulted in murder.

\begin{Shaded}
\begin{Highlighting}[]
\NormalTok{murders\_plot }\OtherTok{\textless{}{-}}\NormalTok{ data\_cln\_y\_m }\SpecialCharTok{\%\textgreater{}\%} \FunctionTok{ggplot}\NormalTok{(}\FunctionTok{aes}\NormalTok{(}\AttributeTok{x =}\NormalTok{ Year\_Month)) }\SpecialCharTok{+} \FunctionTok{geom\_area}\NormalTok{( }\FunctionTok{aes}\NormalTok{(}\AttributeTok{y =}\NormalTok{ Num\_murders), }\AttributeTok{fill=}\StringTok{"red"}\NormalTok{, }\AttributeTok{color =} \DecValTok{1}\NormalTok{, }\AttributeTok{lwd =} \FloatTok{0.5}\NormalTok{, }\AttributeTok{linetype =} \DecValTok{1}\NormalTok{) }\SpecialCharTok{+} \FunctionTok{scale\_x\_date}\NormalTok{(}\AttributeTok{date\_labels =} \StringTok{"\%Y{-}\%b"}\NormalTok{, }\AttributeTok{date\_breaks =} \StringTok{"2 year"}\NormalTok{) }\SpecialCharTok{+} \FunctionTok{scale\_y\_continuous}\NormalTok{(}\AttributeTok{limits =} \FunctionTok{c}\NormalTok{(}\DecValTok{0}\NormalTok{,}\DecValTok{350}\NormalTok{)) }\SpecialCharTok{+} \FunctionTok{theme}\NormalTok{(}\AttributeTok{axis.text.x =} \FunctionTok{element\_text}\NormalTok{(}\AttributeTok{angle =} \DecValTok{45}\NormalTok{, }\AttributeTok{vjust =} \DecValTok{1}\NormalTok{, }\AttributeTok{hjust=}\DecValTok{1}\NormalTok{)) }\SpecialCharTok{+} \FunctionTok{labs}\NormalTok{(}\AttributeTok{title=}\StringTok{"Number of murders per month,}\SpecialCharTok{\textbackslash{}n}\StringTok{2006{-}Jan to 2022{-}Dec"}\NormalTok{, }\AttributeTok{x=} \StringTok{"Year{-}Month"}\NormalTok{, }\AttributeTok{y =} \StringTok{"Number of murders,}\SpecialCharTok{\textbackslash{}n}\StringTok{per month"}\NormalTok{)}

\FunctionTok{plot}\NormalTok{(murders\_plot)}
\end{Highlighting}
\end{Shaded}

\includegraphics{NYPD-shooting-data-project_files/figure-latex/plot_2-1.pdf}

Next is the same graph magnified.

\begin{Shaded}
\begin{Highlighting}[]
\NormalTok{murders\_plot\_zoom }\OtherTok{\textless{}{-}}\NormalTok{ data\_cln\_y\_m }\SpecialCharTok{\%\textgreater{}\%} \FunctionTok{ggplot}\NormalTok{(}\FunctionTok{aes}\NormalTok{(}\AttributeTok{x =}\NormalTok{ Year\_Month)) }\SpecialCharTok{+} \FunctionTok{geom\_area}\NormalTok{( }\FunctionTok{aes}\NormalTok{(}\AttributeTok{y =}\NormalTok{ Num\_murders), }\AttributeTok{fill=}\StringTok{"red"}\NormalTok{, }\AttributeTok{color =} \DecValTok{1}\NormalTok{, }\AttributeTok{lwd =} \FloatTok{0.5}\NormalTok{, }\AttributeTok{linetype =} \DecValTok{1}\NormalTok{) }\SpecialCharTok{+} \FunctionTok{scale\_x\_date}\NormalTok{(}\AttributeTok{date\_labels =} \StringTok{"\%Y{-}\%b"}\NormalTok{, }\AttributeTok{date\_breaks =} \StringTok{"2 year"}\NormalTok{) }\SpecialCharTok{+} \FunctionTok{theme}\NormalTok{(}\AttributeTok{axis.text.x =} \FunctionTok{element\_text}\NormalTok{(}\AttributeTok{angle =} \DecValTok{45}\NormalTok{, }\AttributeTok{vjust =} \DecValTok{1}\NormalTok{, }\AttributeTok{hjust=}\DecValTok{1}\NormalTok{)) }\SpecialCharTok{+} \FunctionTok{labs}\NormalTok{(}\AttributeTok{title=}\StringTok{"Number of murders per month,}\SpecialCharTok{\textbackslash{}n}\StringTok{2006{-}Jan to 2022{-}Dec"}\NormalTok{, }\AttributeTok{x=} \StringTok{"Year{-}Month"}\NormalTok{, }\AttributeTok{y =} \StringTok{"Number of murders,}\SpecialCharTok{\textbackslash{}n}\StringTok{per month"}\NormalTok{)}

\FunctionTok{plot}\NormalTok{(murders\_plot\_zoom)}
\end{Highlighting}
\end{Shaded}

\includegraphics{NYPD-shooting-data-project_files/figure-latex/plot_3-1.pdf}

This plot does show the same pattern in murder rates as shooting
incident rates. You can see the same pattern of deep troughs in winter
months and sharp peaks in summer months. This is not unexpected, as it
is a simple and reasonable jump to think that murder rates would also
follow the same pattern as overall shooting rates. There does seem to be
variability in that similarity however, where some summer months have
sharp drops in murder rates and a given year can have very similar
murder rates month to month, which you can see in 2013. To check and see
if there is a statistically significant correlation between shooting
rates and murder rates, we move onto the second quesition I wish to
answer.

\hypertarget{analysis-of-question-2}{%
\subsubsection{Analysis of Question 2}\label{analysis-of-question-2}}

Below you will see a scatter plot showing murder rates as a function of
shooting incidents, with a liner regression best-fit line running
through the entire plot.

\begin{Shaded}
\begin{Highlighting}[]
\NormalTok{model\_plot }\OtherTok{\textless{}{-}}\NormalTok{ nyc\_final\_data }\SpecialCharTok{\%\textgreater{}\%} \FunctionTok{ggplot}\NormalTok{(}\FunctionTok{aes}\NormalTok{(}\AttributeTok{x =}\NormalTok{ Num\_shootings)) }\SpecialCharTok{+} \FunctionTok{geom\_point}\NormalTok{(}\FunctionTok{aes}\NormalTok{(}\AttributeTok{y=}\NormalTok{Num\_murders)) }\SpecialCharTok{+} \FunctionTok{geom\_line}\NormalTok{(}\FunctionTok{aes}\NormalTok{(}\AttributeTok{y =}\NormalTok{ murders\_pred, }\AttributeTok{color=}\StringTok{"red"}\NormalTok{)) }\SpecialCharTok{+} \FunctionTok{scale\_color\_manual}\NormalTok{(}\AttributeTok{labels =} \FunctionTok{c}\NormalTok{(}\StringTok{"Model"}\NormalTok{), }\AttributeTok{values =} \FunctionTok{c}\NormalTok{(}\StringTok{\textquotesingle{}red\textquotesingle{}}\NormalTok{) ) }\SpecialCharTok{+} \FunctionTok{guides}\NormalTok{(}\AttributeTok{color=}\FunctionTok{guide\_legend}\NormalTok{(}\AttributeTok{title=}\StringTok{""}\NormalTok{) ) }\SpecialCharTok{+} \FunctionTok{ggtitle}\NormalTok{(}\StringTok{"Number of murders per month}\SpecialCharTok{\textbackslash{}n}\StringTok{as a function of total shooting incidents per month"}\NormalTok{) }\SpecialCharTok{+} \FunctionTok{labs}\NormalTok{(}\AttributeTok{x=} \StringTok{"Shooting incidents per month"}\NormalTok{, }\AttributeTok{y =} \StringTok{"Murders per month"}\NormalTok{)}

\FunctionTok{plot}\NormalTok{(model\_plot)}
\end{Highlighting}
\end{Shaded}

\includegraphics{NYPD-shooting-data-project_files/figure-latex/plot_4-1.pdf}

There does appear to be a relationship between shooting rates and murder
rates. It is not a very precise relationship, as you can see that there
is a large spread over which the data falls, but it is possible to see
that the data does cluster along the line drawn by the model. We can
inspect the different statistics generated by the model below.

\begin{Shaded}
\begin{Highlighting}[]
\FunctionTok{summary}\NormalTok{(model)}
\end{Highlighting}
\end{Shaded}

\begin{verbatim}
## 
## Call:
## lm(formula = Num_murders ~ Num_shootings, data = data_cln_y_m)
## 
## Residuals:
##      Min       1Q   Median       3Q      Max 
## -13.8922  -4.0361   0.0418   3.6113  21.0418 
## 
## Coefficients:
##               Estimate Std. Error t value Pr(>|t|)    
## (Intercept)   1.142231   1.150882   0.992    0.322    
## Num_shootings 0.183962   0.008021  22.936   <2e-16 ***
## ---
## Signif. codes:  0 '***' 0.001 '**' 0.01 '*' 0.05 '.' 0.1 ' ' 1
## 
## Residual standard error: 5.898 on 190 degrees of freedom
## Multiple R-squared:  0.7347, Adjusted R-squared:  0.7333 
## F-statistic:   526 on 1 and 190 DF,  p-value: < 2.2e-16
\end{verbatim}

The model determined that it had a multiple R\^{}2 value of 0.73,
indicating that the data doesn't cluster strongly along the best fit
line, This checks out, as we can see in the above graph. The model did
produce a vanishing p-value, however, which does indicate there is a
statistically significant relationship between shooting rates and murder
rates.

However, to check and see if there is any pattern in the residuals of
the model, we should plot those residuals. You will find that plot
below.

\begin{Shaded}
\begin{Highlighting}[]
\NormalTok{model\_res\_plot }\OtherTok{\textless{}{-}} \FunctionTok{ggplot}\NormalTok{(model, }\FunctionTok{aes}\NormalTok{(}\AttributeTok{x =}\NormalTok{ .fitted, }\AttributeTok{y=}\NormalTok{.resid )) }\SpecialCharTok{+} \FunctionTok{geom\_point}\NormalTok{() }\SpecialCharTok{+} \FunctionTok{labs}\NormalTok{(}\AttributeTok{title=}\StringTok{"Residual plot of model"}\NormalTok{, }\AttributeTok{x=} \StringTok{"Predicted murder count"}\NormalTok{, }\AttributeTok{y =} \StringTok{"Residual"}\NormalTok{)}

\FunctionTok{plot}\NormalTok{(model\_res\_plot)}
\end{Highlighting}
\end{Shaded}

\includegraphics{NYPD-shooting-data-project_files/figure-latex/plot_5-1.pdf}

At first glance, the residuals do look sufficiently random to indicate
shooting rates really are a very good predictor of murder rates but if
you start on the left and move right, you can see a slight horn shape to
the scatter plot. In addition, the chart is not evenly random: there is
a clustering of points around murder count = 20. This could indicate
that there are other predictors that are unaccounted for or that the
kind of regression used it not the right kind, that a different form of
regression is needed.

\hypertarget{analysis-of-question-3}{%
\subsubsection{Analysis of Question 3}\label{analysis-of-question-3}}

To give us a quick overview of how each ethnic group is represented in
the data.

Here we have a chart showing normalized total shooting rates per month,
where each racial group in the data is represented as a different color
and takes up a certain percentage of the total shooting incidents for a
given month.

\begin{Shaded}
\begin{Highlighting}[]
\NormalTok{colors\_arr }\OtherTok{\textless{}{-}} \FunctionTok{c}\NormalTok{(}\StringTok{"\#1e5e02"}\NormalTok{, }\StringTok{"\#07cbf2"}\NormalTok{,}\StringTok{"\#4daf4a"}\NormalTok{, }\StringTok{"\#fc0000"}\NormalTok{,}\StringTok{"\#ffff99"}\NormalTok{, }\StringTok{"\#4a3f4f"}\NormalTok{,}\StringTok{"\#fc00f4"}\NormalTok{)}

\NormalTok{chart\_perc\_eth\_all }\OtherTok{\textless{}{-}}\NormalTok{ data\_cln\_eth\_y\_m }\SpecialCharTok{\%\textgreater{}\%} \FunctionTok{ggplot}\NormalTok{(}\FunctionTok{aes}\NormalTok{(}\AttributeTok{x =}\NormalTok{ Year\_Month, }\AttributeTok{y =}\NormalTok{ Num\_shootings\_eth, }\AttributeTok{fill =}\NormalTok{ VIC\_RACE)) }\SpecialCharTok{+} \FunctionTok{geom\_area}\NormalTok{(}\AttributeTok{position =}\StringTok{"fill"}\NormalTok{) }\SpecialCharTok{+} \FunctionTok{scale\_x\_date}\NormalTok{(}\AttributeTok{date\_labels =} \StringTok{"\%Y{-}\%b"}\NormalTok{, }\AttributeTok{date\_breaks =} \StringTok{"2 year"}\NormalTok{) }\SpecialCharTok{+} \FunctionTok{theme}\NormalTok{(}\AttributeTok{axis.text.x =} \FunctionTok{element\_text}\NormalTok{(}\AttributeTok{angle =} \DecValTok{45}\NormalTok{, }\AttributeTok{vjust =} \DecValTok{1}\NormalTok{, }\AttributeTok{hjust=}\DecValTok{1}\NormalTok{)) }\SpecialCharTok{+} \FunctionTok{scale\_y\_continuous}\NormalTok{(}\AttributeTok{labels =}\NormalTok{ scales}\SpecialCharTok{::}\NormalTok{percent) }\SpecialCharTok{+} \FunctionTok{labs}\NormalTok{(}\AttributeTok{title=}\StringTok{"Percent of total shootings per racial group,}\SpecialCharTok{\textbackslash{}n}\StringTok{per month, 2006{-}Jan to 2021{-}Dec"}\NormalTok{, }\AttributeTok{x=} \StringTok{"Year{-}Month"}\NormalTok{, }\AttributeTok{y =} \StringTok{"Percent of total shootings,}\SpecialCharTok{\textbackslash{}n}\StringTok{per racial group, per month"}\NormalTok{)}\SpecialCharTok{+}\FunctionTok{scale\_fill\_manual}\NormalTok{(}\AttributeTok{values=}\NormalTok{colors\_arr) }\SpecialCharTok{+} \FunctionTok{guides}\NormalTok{(}\AttributeTok{fill=}\FunctionTok{guide\_legend}\NormalTok{(}\AttributeTok{title=}\StringTok{"Victim Race"}\NormalTok{))}

\FunctionTok{plot}\NormalTok{(chart\_perc\_eth\_all)}
\end{Highlighting}
\end{Shaded}

\includegraphics{NYPD-shooting-data-project_files/figure-latex/chart per_eth_all-1.pdf}

As you can see, individuals who identity as black make up the
overwhelming majority of victims of shooting violence in NYC over the 15
year time span. Not only are they the ethnic group that is victimized
the most on average over the time investigated, they are the ethnic
group that has been victimized the most every month of the past 15
years, appearing to make up close to the same percentage of targeted
individuals over time.

Below is a chart focusing in on shooting incidents involving black
individuals.

\begin{Shaded}
\begin{Highlighting}[]
\NormalTok{chart\_perc\_eth\_black }\OtherTok{\textless{}{-}}\NormalTok{ data\_cln\_eth\_y\_m[}\FunctionTok{which}\NormalTok{(data\_cln\_eth\_y\_m}\SpecialCharTok{$}\NormalTok{VIC\_RACE }\SpecialCharTok{==}\NormalTok{ ethnicities[}\DecValTok{3}\NormalTok{]),] }\SpecialCharTok{\%\textgreater{}\%} \FunctionTok{ggplot}\NormalTok{(}\FunctionTok{aes}\NormalTok{(}\AttributeTok{x =}\NormalTok{ Year\_Month, }\AttributeTok{fill =}\NormalTok{ VIC\_RACE)) }\SpecialCharTok{+} \FunctionTok{geom\_col}\NormalTok{(}\FunctionTok{aes}\NormalTok{(}\AttributeTok{y =}\NormalTok{ shootings\_perc)) }\SpecialCharTok{+} \FunctionTok{scale\_x\_date}\NormalTok{(}\AttributeTok{date\_labels=}\StringTok{"\%Y{-}\%b"}\NormalTok{,}\AttributeTok{date\_breaks  =}\StringTok{"1 year"}\NormalTok{, }\AttributeTok{limits =} \FunctionTok{as.Date}\NormalTok{(}\FunctionTok{c}\NormalTok{(}\StringTok{\textquotesingle{}2006{-}01{-}01\textquotesingle{}}\NormalTok{,}\StringTok{\textquotesingle{}2021{-}12{-}01\textquotesingle{}}\NormalTok{)) )}\SpecialCharTok{+} \FunctionTok{scale\_y\_continuous}\NormalTok{(}\AttributeTok{labels =}\NormalTok{ scales}\SpecialCharTok{::}\NormalTok{percent) }\SpecialCharTok{+}  \FunctionTok{theme}\NormalTok{(}\AttributeTok{axis.text.x =} \FunctionTok{element\_text}\NormalTok{(}\AttributeTok{angle =} \DecValTok{45}\NormalTok{, }\AttributeTok{vjust =} \DecValTok{1}\NormalTok{, }\AttributeTok{hjust=}\DecValTok{1}\NormalTok{), }\AttributeTok{legend.position =} \StringTok{"top"}\NormalTok{) }\SpecialCharTok{+} \FunctionTok{guides}\NormalTok{(}\AttributeTok{fill=}\FunctionTok{guide\_legend}\NormalTok{(}\AttributeTok{title=}\StringTok{"Victim Race"}\NormalTok{)) }\SpecialCharTok{+} \FunctionTok{labs}\NormalTok{(}\AttributeTok{title=}\StringTok{"Percentage of shootings incidents}\SpecialCharTok{\textbackslash{}n}\StringTok{involving black victims, per month"}\NormalTok{, }\AttributeTok{x=} \StringTok{"Year{-}Month"}\NormalTok{, }\AttributeTok{y =} \StringTok{"Percent of total shootings"}\NormalTok{) }\SpecialCharTok{+} \FunctionTok{scale\_fill\_manual}\NormalTok{(}\AttributeTok{values=}\NormalTok{colors\_arr[}\DecValTok{3}\NormalTok{]) }\SpecialCharTok{+} \FunctionTok{guides}\NormalTok{(}\AttributeTok{fill=}\FunctionTok{guide\_legend}\NormalTok{(}\AttributeTok{title=}\StringTok{"Victim Race"}\NormalTok{))}

\FunctionTok{plot}\NormalTok{(chart\_perc\_eth\_black)}
\end{Highlighting}
\end{Shaded}

\begin{verbatim}
## Warning: Removed 2 rows containing missing values (`geom_col()`).
\end{verbatim}

\includegraphics{NYPD-shooting-data-project_files/figure-latex/chart per_eth_black-1.pdf}

We can see more clearly the month-by-month consistency at which black
individuals are the victims of gun violence. This is illustrated most
clearly by the chart below.

\begin{Shaded}
\begin{Highlighting}[]
\NormalTok{chart\_hist\_perc\_eth\_black }\OtherTok{\textless{}{-}}\NormalTok{ data\_cln\_eth\_y\_m[}\FunctionTok{which}\NormalTok{(data\_cln\_eth\_y\_m}\SpecialCharTok{$}\NormalTok{VIC\_RACE }\SpecialCharTok{==}\NormalTok{ ethnicities[}\DecValTok{3}\NormalTok{]),] }\SpecialCharTok{\%\textgreater{}\%} \FunctionTok{ggplot}\NormalTok{(}\FunctionTok{aes}\NormalTok{( }\AttributeTok{x =}\NormalTok{ shootings\_perc)) }\SpecialCharTok{+} \FunctionTok{geom\_histogram}\NormalTok{(}\FunctionTok{aes}\NormalTok{(}\AttributeTok{fill =}\NormalTok{ colors\_arr[}\DecValTok{3}\NormalTok{]), }\AttributeTok{bins=}\DecValTok{30}\NormalTok{) }\SpecialCharTok{+} \FunctionTok{scale\_x\_continuous}\NormalTok{(}\AttributeTok{labels =}\NormalTok{ scales}\SpecialCharTok{::}\NormalTok{percent, }\AttributeTok{limits =} \FunctionTok{c}\NormalTok{(}\DecValTok{0}\NormalTok{,}\DecValTok{1}\NormalTok{)) }\SpecialCharTok{+}  \FunctionTok{theme}\NormalTok{(}\AttributeTok{legend.position =} \StringTok{"top"}\NormalTok{) }\SpecialCharTok{+} \FunctionTok{labs}\NormalTok{(}\AttributeTok{title=}\StringTok{"Histogram of percentage of total shootings involving black victims"}\NormalTok{, }\AttributeTok{x=} \StringTok{"Percent of total shootings where vitcim is black"}\NormalTok{, }\AttributeTok{y =} \StringTok{"Count"}\NormalTok{) }\SpecialCharTok{+} \FunctionTok{scale\_fill\_manual}\NormalTok{(}\AttributeTok{values=}\NormalTok{colors\_arr[}\DecValTok{3}\NormalTok{], }\AttributeTok{labels=}\FunctionTok{c}\NormalTok{(}\StringTok{\textquotesingle{}BLACK\textquotesingle{}}\NormalTok{)) }\SpecialCharTok{+} \FunctionTok{guides}\NormalTok{(}\AttributeTok{fill=}\FunctionTok{guide\_legend}\NormalTok{(}\AttributeTok{title=}\StringTok{"Victim Race"}\NormalTok{)) }

\FunctionTok{plot}\NormalTok{(chart\_hist\_perc\_eth\_black)}
\end{Highlighting}
\end{Shaded}

\begin{verbatim}
## Warning: Removed 2 rows containing missing values (`geom_bar()`).
\end{verbatim}

\includegraphics{NYPD-shooting-data-project_files/figure-latex/unnamed-chunk-2-1.pdf}

With this graph, you can see that number of shooting incidents each
month involving black individuals make up 60\% to 80\% of the total
shooting instances in NYC over the span of data collection. Below is
another chart showing the average rates of shooting incidents involving
each ethnic group.

\begin{Shaded}
\begin{Highlighting}[]
\NormalTok{shooting\_perc\_plot }\OtherTok{\textless{}{-}}\NormalTok{ data\_cln\_eth\_tots }\SpecialCharTok{\%\textgreater{}\%} \FunctionTok{ggplot}\NormalTok{(}\FunctionTok{aes}\NormalTok{(}\AttributeTok{x =}\NormalTok{ VIC\_RACE, }\AttributeTok{y =}\NormalTok{ shootings\_perc, }\AttributeTok{fill =}\NormalTok{ VIC\_RACE)) }\SpecialCharTok{+} \FunctionTok{geom\_col}\NormalTok{() }\SpecialCharTok{+} \FunctionTok{geom\_text}\NormalTok{(}\FunctionTok{aes}\NormalTok{(}\AttributeTok{label =} \FunctionTok{paste0}\NormalTok{(}\FunctionTok{round}\NormalTok{(shootings\_perc}\SpecialCharTok{*}\DecValTok{100}\NormalTok{, }\DecValTok{2}\NormalTok{), }\StringTok{"\%"}\NormalTok{), }\AttributeTok{y =}\NormalTok{ shootings\_perc }\SpecialCharTok{+} \FloatTok{0.02}\NormalTok{)) }\SpecialCharTok{+} \FunctionTok{scale\_y\_continuous}\NormalTok{(}\AttributeTok{labels =}\NormalTok{ scales}\SpecialCharTok{::}\NormalTok{percent) }\SpecialCharTok{+}  \FunctionTok{theme}\NormalTok{(}\AttributeTok{axis.text.x =} \FunctionTok{element\_blank}\NormalTok{()) }\SpecialCharTok{+} \FunctionTok{scale\_fill\_manual}\NormalTok{(}\AttributeTok{values=}\NormalTok{colors\_arr)  }\SpecialCharTok{+} \FunctionTok{labs}\NormalTok{(}\AttributeTok{title=}\StringTok{"Percentage of shooting incidents per racial group,}\SpecialCharTok{\textbackslash{}n}\StringTok{2006 to 2022"}\NormalTok{, }\AttributeTok{x=} \StringTok{"Victim Race"}\NormalTok{, }\AttributeTok{y =} \StringTok{"Percent composition of total}\SpecialCharTok{\textbackslash{}n}\StringTok{shooting incidents"}\NormalTok{) }\SpecialCharTok{+} \FunctionTok{guides}\NormalTok{(}\AttributeTok{fill=}\FunctionTok{guide\_legend}\NormalTok{(}\AttributeTok{title=}\StringTok{"Victim Race"}\NormalTok{))}

\FunctionTok{plot}\NormalTok{(shooting\_perc\_plot)}
\end{Highlighting}
\end{Shaded}

\includegraphics{NYPD-shooting-data-project_files/figure-latex/chart shooting_perc_plot-1.pdf}

Once again, it is clear to see that black individuals are overwhelmingly
the most victimized racial group. From the earlier investigation into
whether shooting rates predict murder rates, we can guess that black
individuals are also a largely present in murder cases. Lets look at the
next chart to see if that is the case.

\begin{Shaded}
\begin{Highlighting}[]
\NormalTok{chart\_perc\_eth\_black\_mur }\OtherTok{\textless{}{-}}\NormalTok{ data\_cln\_eth\_y\_m[}\FunctionTok{which}\NormalTok{(data\_cln\_eth\_y\_m}\SpecialCharTok{$}\NormalTok{VIC\_RACE }\SpecialCharTok{==}\NormalTok{ ethnicities[}\DecValTok{3}\NormalTok{]),] }\SpecialCharTok{\%\textgreater{}\%} \FunctionTok{ggplot}\NormalTok{(}\FunctionTok{aes}\NormalTok{(}\AttributeTok{x =}\NormalTok{ Year\_Month, }\AttributeTok{fill =}\NormalTok{ VIC\_RACE)) }\SpecialCharTok{+} \FunctionTok{geom\_col}\NormalTok{(}\FunctionTok{aes}\NormalTok{(}\AttributeTok{y =}\NormalTok{ murders\_perc)) }\SpecialCharTok{+} \FunctionTok{scale\_x\_date}\NormalTok{(}\AttributeTok{date\_labels=}\StringTok{"\%Y{-}\%b"}\NormalTok{,}\AttributeTok{date\_breaks  =}\StringTok{"1 year"}\NormalTok{, }\AttributeTok{limits =} \FunctionTok{as.Date}\NormalTok{(}\FunctionTok{c}\NormalTok{(}\StringTok{\textquotesingle{}2006{-}01{-}01\textquotesingle{}}\NormalTok{,}\StringTok{\textquotesingle{}2021{-}12{-}01\textquotesingle{}}\NormalTok{)) )}\SpecialCharTok{+} \FunctionTok{scale\_y\_continuous}\NormalTok{(}\AttributeTok{labels =}\NormalTok{ scales}\SpecialCharTok{::}\NormalTok{percent) }\SpecialCharTok{+}  \FunctionTok{theme}\NormalTok{(}\AttributeTok{axis.text.x =} \FunctionTok{element\_text}\NormalTok{(}\AttributeTok{angle =} \DecValTok{45}\NormalTok{, }\AttributeTok{vjust =} \DecValTok{1}\NormalTok{, }\AttributeTok{hjust=}\DecValTok{1}\NormalTok{), }\AttributeTok{legend.position =} \StringTok{"top"}\NormalTok{) }\SpecialCharTok{+} \FunctionTok{guides}\NormalTok{(}\AttributeTok{fill=}\FunctionTok{guide\_legend}\NormalTok{(}\AttributeTok{title=}\StringTok{"Victim Race"}\NormalTok{)) }\SpecialCharTok{+} \FunctionTok{labs}\NormalTok{(}\AttributeTok{title=}\StringTok{"Number of murder per month where victim is black"}\NormalTok{, }\AttributeTok{x=} \StringTok{"Year{-}Month"}\NormalTok{, }\AttributeTok{y =} \StringTok{"Number of shooting incidents,}\SpecialCharTok{\textbackslash{}n}\StringTok{per month"}\NormalTok{) }\SpecialCharTok{+} \FunctionTok{scale\_fill\_manual}\NormalTok{(}\AttributeTok{values=}\NormalTok{colors\_arr[}\DecValTok{3}\NormalTok{]) }\SpecialCharTok{+} \FunctionTok{guides}\NormalTok{(}\AttributeTok{fill=}\FunctionTok{guide\_legend}\NormalTok{(}\AttributeTok{title=}\StringTok{"Victim Race"}\NormalTok{))}

\FunctionTok{plot}\NormalTok{(chart\_perc\_eth\_black\_mur)}
\end{Highlighting}
\end{Shaded}

\begin{verbatim}
## Warning: Removed 2 rows containing missing values (`geom_col()`).
\end{verbatim}

\includegraphics{NYPD-shooting-data-project_files/figure-latex/chart per_eth_black_mur-1.pdf}

It is clear from this chart that, yes, black individuals are also the
most murdered racial group in the data set. As there are many high and
low rates in this graph, it will be more helpful to view this data in
the form of a histogram.

\begin{Shaded}
\begin{Highlighting}[]
\NormalTok{chart\_hist\_perc\_eth\_black\_mur }\OtherTok{\textless{}{-}}\NormalTok{ data\_cln\_eth\_y\_m[}\FunctionTok{which}\NormalTok{(data\_cln\_eth\_y\_m}\SpecialCharTok{$}\NormalTok{VIC\_RACE }\SpecialCharTok{==}\NormalTok{ ethnicities[}\DecValTok{3}\NormalTok{]),] }\SpecialCharTok{\%\textgreater{}\%} \FunctionTok{ggplot}\NormalTok{(}\FunctionTok{aes}\NormalTok{( }\AttributeTok{x =}\NormalTok{ murders\_perc)) }\SpecialCharTok{+} \FunctionTok{geom\_histogram}\NormalTok{(}\FunctionTok{aes}\NormalTok{(}\AttributeTok{fill =}\NormalTok{ colors\_arr[}\DecValTok{3}\NormalTok{]),}\AttributeTok{bins =} \DecValTok{20}\NormalTok{) }\SpecialCharTok{+} \FunctionTok{scale\_x\_continuous}\NormalTok{(}\AttributeTok{labels =}\NormalTok{ scales}\SpecialCharTok{::}\NormalTok{percent, }\AttributeTok{limits =} \FunctionTok{c}\NormalTok{(}\DecValTok{0}\NormalTok{,}\DecValTok{1}\NormalTok{)) }\SpecialCharTok{+}  \FunctionTok{theme}\NormalTok{(}\AttributeTok{legend.position =} \StringTok{"top"}\NormalTok{) }\SpecialCharTok{+} \FunctionTok{labs}\NormalTok{(}\AttributeTok{title=}\StringTok{"Histogram of percentage of total murders involving black victims"}\NormalTok{, }\AttributeTok{x=} \StringTok{"Percent of total murders where vitcim is black"}\NormalTok{, }\AttributeTok{y =} \StringTok{"Count"}\NormalTok{) }\SpecialCharTok{+} \FunctionTok{scale\_fill\_manual}\NormalTok{(}\AttributeTok{values=}\NormalTok{colors\_arr[}\DecValTok{3}\NormalTok{], }\AttributeTok{labels=}\FunctionTok{c}\NormalTok{(}\StringTok{\textquotesingle{}BLACK\textquotesingle{}}\NormalTok{)) }\SpecialCharTok{+} \FunctionTok{guides}\NormalTok{(}\AttributeTok{fill=}\FunctionTok{guide\_legend}\NormalTok{(}\AttributeTok{title=}\StringTok{"Victim Race"}\NormalTok{)) }

\FunctionTok{plot}\NormalTok{(chart\_hist\_perc\_eth\_black\_mur)}
\end{Highlighting}
\end{Shaded}

\begin{verbatim}
## Warning: Removed 2 rows containing missing values (`geom_bar()`).
\end{verbatim}

\includegraphics{NYPD-shooting-data-project_files/figure-latex/unnamed-chunk-3-1.pdf}

As you can see, the distribution of murder rates is greater than for
shooting rates, but with an average and median rate that falls at a
similar percentage as for shooting rates. Lets inspect the final graphof
average murder rates involving each ethnic group.

\begin{Shaded}
\begin{Highlighting}[]
\NormalTok{murder\_perc\_plot }\OtherTok{\textless{}{-}}\NormalTok{ data\_cln\_eth\_tots }\SpecialCharTok{\%\textgreater{}\%} \FunctionTok{ggplot}\NormalTok{(}\FunctionTok{aes}\NormalTok{(}\AttributeTok{x =}\NormalTok{ VIC\_RACE, }\AttributeTok{y =}\NormalTok{ murders\_perc, }\AttributeTok{fill =}\NormalTok{ VIC\_RACE)) }\SpecialCharTok{+} \FunctionTok{geom\_col}\NormalTok{() }\SpecialCharTok{+} \FunctionTok{geom\_text}\NormalTok{(}\FunctionTok{aes}\NormalTok{(}\AttributeTok{label =} \FunctionTok{paste0}\NormalTok{(}\FunctionTok{round}\NormalTok{(murders\_perc}\SpecialCharTok{*}\DecValTok{100}\NormalTok{, }\DecValTok{2}\NormalTok{), }\StringTok{"\%"}\NormalTok{), }\AttributeTok{y =}\NormalTok{ murders\_perc }\SpecialCharTok{+} \FloatTok{0.02}\NormalTok{)) }\SpecialCharTok{+} \FunctionTok{scale\_y\_continuous}\NormalTok{(}\AttributeTok{labels =}\NormalTok{ scales}\SpecialCharTok{::}\NormalTok{percent) }\SpecialCharTok{+}  \FunctionTok{theme}\NormalTok{(}\AttributeTok{axis.text.x =} \FunctionTok{element\_blank}\NormalTok{()) }\SpecialCharTok{+} \FunctionTok{scale\_fill\_manual}\NormalTok{(}\AttributeTok{values=}\NormalTok{colors\_arr) }\SpecialCharTok{+} \FunctionTok{labs}\NormalTok{(}\AttributeTok{title=}\StringTok{"Percentage of murders per racial group,}\SpecialCharTok{\textbackslash{}n}\StringTok{2006 to 2022"}\NormalTok{, }\AttributeTok{x=} \StringTok{"Victim Race"}\NormalTok{, }\AttributeTok{y =} \StringTok{"Percent composition of total murders"}\NormalTok{) }\SpecialCharTok{+} \FunctionTok{guides}\NormalTok{(}\AttributeTok{fill=}\FunctionTok{guide\_legend}\NormalTok{(}\AttributeTok{title=}\StringTok{"Victim Race"}\NormalTok{))}

\FunctionTok{plot}\NormalTok{(murder\_perc\_plot)}
\end{Highlighting}
\end{Shaded}

\includegraphics{NYPD-shooting-data-project_files/figure-latex/chart murder_perc_plot-1.pdf}

From all of these charts you can see that people who identify or had
been identified as black are by far the most targeted ethnic group
historically. In addition, they constantly make up around 3/4 of the
total number of shooting incidents in NYC, fluctuating between just
above 50\% to over 80\% of the total shooting incidents.

\hypertarget{conclusion}{%
\subsection{Conclusion}\label{conclusion}}

In conclusion, investigating of the data set to answer my questions
revealed that

\begin{enumerate}
\def\labelenumi{\arabic{enumi}.}
\item
  There was a consistent cyclical pattern to the rates of shooting rates
  and murder rates in NYC, where rates were highest in the summer
  months, while rates were lowest in the winter monsths,.
\item
  Shooting rates by month are a statistically significant predictor of
  murder rates in NYC, but they are not a great predictor and there
  could be other factors not accounted for that influence murder rates
\item
  Black individuals are always the most frequent victims of gun violence
  in NYC. In addition, they on average make up about 70\% of a given
  months total victims, and always make up more than 50\% of a given
  months victims.
\end{enumerate}

A quick google search will tell you that from U.S. census data, the
black population of NYC makes up only about 23\% of the total
population, yet we see here that black individuals in NYC are the
victims of gun violence almost 75\% percent of the time. A very large
and concerning discrepancy.

\end{document}
